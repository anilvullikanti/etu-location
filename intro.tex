
\red{make text more current}

The 2014 Ebola outbreak was the largest in history, accounting for more than 11,300
deaths in West Africa. Starting in Guinea in December 2013 \cite{Baize2014}, 
the epidemic quickly spread to the neighboring countries,
and sustained autochthonous transmission was seen in Liberia and Sierra Leone
by March, 2014.  The health infrastructure in the West African countries of Guinea, Sierra Leone and Liberia
was severely strained by this outbreak. Patients had to travel long distances
by various modes of transport to reach a hospital, or worse, a holding center
before they could be transferred to a hospital,\footnote{%
\url{http://www.cdc.gov/mmwr/preview/mmwrhtml/mm6439a4.htm}}
which played a big role in increasing transmission in the initial stages of the outbreak.
As part of the response efforts,
Ebola Treatment Centers (ETCs) were set up with international help
in order to isolate and treat infected individuals.
However, this was done without necessarily
optimizing access to healthcare for infected patients, so some of the ETCs
were underutilized.  For instance, several
ETCs never saw any patients.\footnote{%
\url{http://www.nytimes.com/2015/04/12/world/africa/idle-ebola-clinics-in-liberia-are-seen-as-misstep-in-us-relief-effort.html?_r=0}}.
While there might be a multitude of reasons (which are still being studied by
the public health community), some of these include: limitations in forecasting
Ebola spread (including lack of data and behavioral adaptations at all levels), 
delays in response, and the disconnect between logistical planning and forecasting.
Some of the key problems that arose in that process are: \emph{How should
such facilities be deployed? What objectives should be considered in their planning,
and what information is needed to ensure effective deployment?}

These problems are not restricted to the Ebola outbreak, but arise more generally whenever
the health infrastructure gets strained, which could happen during a pandemic flu scenario.
In this paper, we study problems of optimizing the deployment of healthcare resources including:
which objectives are important, how to deploy facilities in the initial stages of the outbreak
without complete information, and how to plan incremental deployment, when the number
of facilities is not known ahead of time. We develop novel methods for incremental facility
location, which give solutions with provably low cost, relative to the optimum solution
(as discussed later), and evaluate them using detailed simulated outbreaks of
Ebola in West Africa.

As we may expect, we find that the optimum solution depends crucially on the specific objective
(e.g., individual vs social cost), and it is more effective to plan based on forecasted demands,
instead of just the static population demand. More surprisingly, we show that
there exists a natural notion of a \emph{kernel}, such that an efficient 
\emph{near-optimal} solution can be obtained by:
(1) opening facilities initially at the kernel locations, and
(2) opening subsequent facilities incrementally using a novel method based on linear programming.
This is particularly useful if the number of facilities to
be deployed is not known in advance, or they have to be deployed incrementally.
Thus, our results suggest a simple and efficient method for deploying healthcare resources.

\textcolor{red}{Bryan: Some discussion of healthcare logistics research}

%\noindent
%\textbf{Summary of our contributions}
%\begin{enumerate}
%\item
%Individual vs social cost: the optimal solution depends crucially on what
%kind of objective is being considered.
%\item
%Need for spatio-temporal forecasting of epidemic spread: we find that
%using forecasted incidence rates can lead to
%close to 10-15\% reduction in the average travel time. Even with x\% errors
%in the forecasts, we get y\% improvement in the individual objective.
%\item
%Incremental facility location: we develop a novel method \textsc{OnlinekMed}, which
%deploys facilities incrementally, based on the new infections (without altering
%the ones already deployed). We find it is x\% worse compared to the extreme
%approach that allows facilities to be redeployed completely, based on the observed
%infection rates.
%\item
%Kernel facilities: if the number of facilities ($k$) is not known in advance,
%we find that there exist certain locations, such that deployment at these locations
%remains efficient for any $k$. 
%\end{enumerate}

